\documentclass{article}
\usepackage{amsmath, amsfonts, amssymb}
\usepackage{graphicx}

\graphicspath{{Images/}}

\title{Resultados c\'alculo de tamaños en los poros de canales i\'onicos.}
\author{Jonathan Cotrino Lemus}
\date{28 de Abril de 2017}


\begin{document}

\begin{center}
\textbf{\large{Resultados C\'alculo del Tamaño en los Poros de Canales I\'onicos}}
\end{center}

\section{Introducci\'on}

En este documento se presentan los resultados obtenidos al calcular computacionalmente, a trav\'es de m\'etodo Makov Chain Monte Carlo (con el algor\'itmo Metropolis-Hasting), del tamaño de los poros de dos canales i\'onicos con base en algunos puntos de los mismos.

\section{Resultados}

Los datos que se toman como base para determinar el radio, y por lo tanto el tamaño del  poro del primer canal i\'onico, es pueden observar en la figura 1. De forma correspondiente, en  la figura 2 se pueden observar datos correspondientes al poro del segundo canal i\'onico.

\begin{figure}
\begin{center}
\includegraphics[scale=0.2]{Figure_1}
\end{center}
\caption{Puntos correspondientes al poro de el primer canal i\'onico.}
\end{figure}
\begin{figure}


\begin{center}
\includegraphics[scale=0.2]{Figure_2}
\end{center}
\caption{Puntos correspondiente al poro del segundo canal i\'onico.}
\end{figure}

En las dem\'as gr\'aficas se observan los resultados sucesivos del c\'alculo de radio m\'aximo para cada iteraci\'on en el programa. Podemos observar c\'omo es que a medida que se itera, el radio aumenta conforme al mejoramiento del punto tomado. 

\begin{figure}
\begin{center}
\includegraphics[scale=0.2]{Figure_3}
\end{center}
\caption{Primera iteraci\'on.}
\end{figure}

\begin{figure}
\begin{center}
\includegraphics[scale=0.2]{Figure_4}
\end{center}
\caption{Segunda iteraci\'on.}
\end{figure}

\begin{figure}
\begin{center}
\includegraphics[scale=0.2]{Figure_5}
\end{center}
\caption{Tercera iteraci\'on.}
\end{figure}


\begin{figure}
\begin{center}
\includegraphics[scale=0.2]{Figure_6}
\end{center}
\caption{Sexta iteraci\'on.}
\end{figure}


\begin{figure}
\begin{center}
\includegraphics[scale=0.2]{Figure_7}
\end{center}
\caption{Septima iteraci\'on.}
\end{figure}


\begin{figure}
\begin{center}
\includegraphics[scale=0.2]{Figure_8}
\end{center}
\caption{Octava iteraci\'on.}
\end{figure}



\end{document}
